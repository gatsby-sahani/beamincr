\documentclass[\jobname]{beamer} % use jobname="handout" for handout
\usepackage{beamincr}

\begin{document}

\newcounter{exampc}
\newenvironment{examp}
               {\stepcounter{exampc}%
               \begin{frame}{example \theexampc}}
               {\end{frame}}

               
\begin{examp}
  text on slides 1 and up\\
  \onslide<2->
  text on slides 2 and up\\
  \onslide<3-4>{
  text only on slides 3 and 4\\
  } 
  \only<5>{text only on slide 5\\}
  more text on slides 2 and up\\
\end{examp}

\begin{examp}
  \resetincr % not standard BEAMER
  text on slides 1+\\
  \onslide<+->
  text on slides 2+\\
  \onslide<+-+(1)>{ % increments counter by 1, despite the two +s
  text only on slides 3-4\\
  } 
  \onslide<+>{} % increment counter by another
  \only<+>{text only on slide 5\\}
  more text on slides 2+\\
\end{examp}

\begin{examp}
  \resetincr % not standard BEAMER
  text on slide 1+\\
  \onslide<3>{text on slide 3}\\
  text on slide 1+\\
  \onslide<+->
  text on slide 2+\\
  \onslide<4->
  text on slide 4+\\ % increment number is still 2!
  \onslide<+->
  text on slide 3+\\
\end{examp}

\begin{examp}
  \resetincr
  \begin{center}
    Two lists \onslide<+->{in sync}
  \end{center}
  \begin{columns}
    \begin{column}{.2\textwidth}
      \begin{itemize}[<+-| alert@+>]
      \item Apple \item Peach \item Plum \item Orange
      \end{itemize}
    \end{column}
    \begin{column}{.2\textwidth}
      \resetincr[2]
      \begin{itemize}[<+-| alert@+>]
      \item green \item peach \item purple \item orange
      \end{itemize}
    \end{column}
  \end{columns}
\end{examp}


\begin{examp}
  text on slides 1-\\
  \onslide<+->
  text still on slides 1-\\
  \onslide<+->
  text on slides 2-\\
  \resetincr\onslide<.->
  text on slides 1-\\
  \onslide<+->
  text on slides 2-
\end{examp}

\stepcounter{exampc}
\begin{frame}[label=twolists]{examp \theexampc}
  \resetincr
  \begin{center}
    Two lists \onslide<+->{in sync}\\
    \onslide<+->{with more material\\}
    \onslide<+->{at the top}
  \end{center}
  \begin{columns}
    \begin{column}{.2\textwidth}
      \incrlabel{startlist}%
      \begin{itemize}[<+-| alert@+>]
      \item Apple \item Peach \incrlabel{halfway} \item Plum \item Orange
      \end{itemize}
    \end{column}
    \begin{column}{.2\textwidth}
      \resetincr[/startlist/]%
      \begin{itemize}[<+-| alert@+>]
      \item green \item peach \item purple \item orange
      \end{itemize}
    \end{column}
  \end{columns}
  \vfill
  \onslide<+->
  The final increment is \incrref{.}. 
  \incrlabel{end}
\end{frame}

\againframe<1,/halfway/,/end(-1)/-/end/>{twolists}

\begin{examp}
  \resetincr[2]
  \incrlabel{two}
  \parseincrrefspec{/two/-/two(3)/} % prints 2-5
\end{examp}

\begin{frame}<1>[label=test]
  \resetincr
  \onslide<+-> \incrlabel{two}
  2+
  \onslide<+->
  3+
\end{frame}

\againframe</two/>{test}

\newtoks\mymatrix
\begin{examp}
  \resetincr
  \begin{incralign*}[<+->&<.->&<+->&<.->&<+->&<.->] % both sides of up to 3 eqns/line appear at once
    x\incrlabel{x} &= y & 1 &{}< 2 \\
    </x/-> x^2 &= y^2 & </x/->\resetincr[/x/] e^{i\pi} &<+->= -1 \\
    \sum_n f(n) &\to \int f(x) dx
  \end{incralign*}

  (increment at this line = \incrref{.})\\
  \mymatrix={\begin{pmatrix} 1 & 2 \\ \alt<+->{3}{2} & 4 \\ \end{pmatrix}}
  \begin{incralign}
    \incrlabel{mat}\the\mymatrix \resetincr[/mat/] & \text{is \only<+->{not }singular}
  \end{incralign}

\end{examp}


\end{document}
