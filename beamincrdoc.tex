\documentclass[a4paper]{ltxdoc}
% \usepackage[margin=2cm]{geometry}
%% Fonts, etc.

%% Language and font encodings
\usepackage[english]{babel}
%\usepackage[utf8x]{inputenc}
%\usepackage[T1]{fontenc}        
% \usepackage{microtype}      % fine pdf font control

% \usepackage[scaled=0.92]{helvet} \renewcommand{\familydefault}{\sfdefault}
% \usepackage{xcolor} -- loaded by tikz
% \usepackage{soul} % \ul \st \hl

%% packages from beamer user guide
\usepackage{cmap}
\usepackage{lmodern}
\usepackage[T1]{fontenc}
\usepackage[utf8]{inputenc}
\usepackage{amsmath,amssymb}
\usepackage{pifont}
\usepackage{makeidx}
\usepackage{pgf,xcolor}
\usepackage[pdfborder={0 0 0},bookmarksnumbered]{hyperref}
\usepackage[left=2.25cm,right=2.25cm,top=2.5cm,bottom=2.5cm,nohead]{geometry}
\usepackage{translator}

%% beamer userguide macros
\input{beamerug-macros}

%% Maths
\usepackage{amssymb,amsmath}
% \usepackage{algorithm,algpseudocode}
% \usepackage{mathtools}
% \usepackage{mathalfa}
% \usepackage{bbm}
% \usepackage{mathrsfs}
% \usepackage{gensymb}
% \input m.macros
% \input m.symbols

%% Other typsetting
% \usepackage{booktabs}       % professional-quality tables
% \usepackage{nicefrac}       % compact symbols for 1/2, etc.
% \usepackage{enumerate}

%% Graphics
% \usepackage{graphicx,tikz}
% \usetikzlibrary{positioning}
% \usetikzlibrary{fit}
% \usetikzlibrary{calc}

%% Floats
% \usepackage{wrapfig}

%% Margin bubbles
% \usepackage{mbubble}
% \mbubblestyle{ms}{font=\tiny,label=ms,color=blue}
% \newcommand{\comment}[1]{}

%% Sections
% \usepackage{titlesec} % format section titles
% \titleformat{command}[shape]{format}{label}{sep}{before-code}[after-code]

% \usepackage{caption,subcaption}
% \DeclareCaptionSubType[Alph]{figure}
% \captionsetup[subfigure]{labelformat=simple,labelfont={sf,bf,large},textfont={sf}}
% \renewcommand{\thesubfigure}{{\sffamily\alph{subfigure}}}

%% Spacing
% \usepackage[doublespacing]{setspace}
% \parindent 0pt
% \parskip 1ex

%% Citations, refs, hyperlinks
% \usepackage[colorlinks=true, allcolors=blue]{hyperref}
% \usepackage{natbib}
\usepackage[capitalize]{cleveref}
% \crefname{equation}{}{}
% \crefname{Equation}{Equation}{Equations}



%% Title

\title{Enhanced \beamer\ increments:\\
the \texttt{beamincr} package}
\author{Maneesh Sahani}
\date{\today}
% \usepackage{filemod} % access to file modification times
% \date{\Filemodtoday{\jobname}}  % replace \jobname if compiling from wrapper

% \usepackage{titling} % modify title(page) appearance
%\pretitle{\begin{center}\LARGE}
%\posttitle{\par\end{center}\vskip 0.5em}
%\preauthor{\begin{center} \large \lineskip 0.5em \begin{tabular}[t]{c}}
%\postauthor{\end{tabular}\par\end{center}}
%\predate{\begin{center}\large}
%\postdate{\par\end{center}}



%% Document


\begin{document}

\maketitle

\noindent
The \beamer\ class provides a number of powerful mechanisms to compose
and control ``overlays'' or ``animations,'' where a single complex
slide is built up (or changed) piece by piece.
%
While powerful, these mechanisms may sometimes be usefully extended.  This
package provides some such enhancements.

\section{Background: overlays and increments}

The basic \beamer\ display unit is the \texttt{frame}.  A
frame may be rendered step-by-step, in which case the individual
versions of the frame are called ``overlays'' or ``slides''.  We will
use these terms interchangeably.
%
\beamer\ allows you to place material on an arbitrary slide in a frame like this
\example
\begin{verbatim}
\begin{frame}
  text on slides 1 and up\\
  \onslide<2->
  text on slides 2 and up\\
  \onslide<3-4>{
  text only on slides 3 and 4\\
  } 
  \only<5>{text only on slide 5\\}
  more text on slides 2 and up\\
\end{frame}
\end{verbatim}
You can read about the differences between \myprintcommand{onslide}
and \myprintcommand{only}, and the many other overlay-sensitive
commands, in the \beamer\ user guide.  Note in particular the
difference between the argument form and the declaration forms of
\myprintcommand{onslide}.  \myprintcommand{only} only works with an
argument.

This explicit numbering approach becomes burdensome when you want many overlays.
You have to keep track of the numbers explicitly, and if you subsequently add a
step early in the sequence you need to re-number the rest.  Thus, \beamer\ also
provides an incremental overlay specification.  The following code will produce
the same effect as that above.  \example
\begin{verbatim}
\begin{frame}
  \resetincr % not standard BEAMER
  text on slides 1+\\
  \onslide<+->
  text on slides 2+\\
  \onslide<+-+(1)>{ % increments counter by 1, despite the two +s
  text only on slides 3-4\\
  } 
  \onslide<+>{} % increment counter by another
  \only<+>{text only on slide 5\\}
  more text on slides 2+\\
\end{frame}
\end{verbatim}
This form allows easy automation using default overlay
specificiations.  For instance (from the \beamer\ user guide)
\example
\begin{verbatim}
\begin{itemize}[<+-| alert@+>]
\item Apple
\item Peach
\item Plum
\item Orange
\end{itemize}
\end{verbatim}
There are important and sometimes not-entirely-intuitive differences between the
incremental and explicit numbering systems.  So we will refer to the steps
implied in this way as ``increments''.  They will mostly match slide numbers,
but not always, as this example shows: \example
\begin{verbatim}
\begin{frame}
  \resetincr % not standard BEAMER
  text on slide 1+\\
  \onslide<3>{text on slide 3}\\
  text on slide 1+\\
  \onslide<+->
  text on slide 2+\\
  \onslide<4->
  text on slide 4+\\ % increment number is still 2!
  \onslide<+->
  text on slide 3+\\
\end{frame}
\end{verbatim}
The increments have their own internal logic (specifically, their own internal
counter) which is not affected by any explicit slide specifications that may
appear in-between incremental calls.  It may make sense to think of the
increment number as being associated with \emph{where} in the source file the
material appears, rather than (necessarily) on \emph{which slide} it appears.

There are a couple of oddities with the way increments work that often trip up
first-time users.  There are also some extensions that would be nice, like the
ability to refer to a specific increment elsewhere in the frame.  These things
are certainly possible in stock \beamer, but take some digging into internals.
The tools here make things a bit easier.

As an aside, \beamer\ has another incremental overlay system based on the
|\pause| command.  The |\pause| command uses the same counter as increments, but
inteprets it slightly differently. This difference is discussed in 
\cref{sec:internals}. As a result, the two sets of specifications don't play very
well together, at least from the viewpoint of non-experts.  More on this below.
I strongly suggest avoiding \myprintcommand{pause} entirely.


\section{Setting increments}\label{sec:setting}

\begin{command}{\resetincr\oarg{incr}}
  This command resets the increment number to 1, or to the value defined by
  \meta{incr} if given.  It doesn't directly affect the slide on which
  subsequent text appears, but it does change effect of subsequent |<+>| or
  |<.>| increments.  The command may be useful to synchronise overlays in (say)
  two columns or between highlighted bullet points and highlighting in a figure.
  \example
\begin{verbatim}
\begin{frame}
  \resetincr
  \begin{center}
    Two lists \onslide<+->{in sync}
  \end{center}
  \begin{columns}
    \begin{column}{.2\textwidth}
      \begin{itemize}[<+-| alert@+>]
      \item Apple \item Peach \item Plum \item Orange
      \end{itemize}
    \end{column}
    \begin{column}{.2\textwidth}
      \resetincr[2] % restart the increment counter to sync
      \begin{itemize}[<+-| alert@+>]
      \item green \item yellow \item purple \item orange
      \end{itemize}
    \end{column}
  \end{columns}
\end{frame}
\end{verbatim}
The argument \meta{incr} must either be a number or be an increment
specification enclosed in |/ /| (these are defined in \cref{sec:labels}).  It cannot specify any
sort of range, or be |+| or |.|, although |/./| and things like |/.(2)/| are
allowed.

It is useful to call |\resetincr| at the start of every increment-based slide
(as we have in the examples here).  This avoids some potentially confusing
behaviour that comes from the way the increment counter is implemented in
\beamer:
\example
\begin{verbatim}
\begin{frame}
  text on slides 1-\\
  \onslide<+->
  text still on slides 1-\\
  \onslide<+->
  text on slides 2-
  \resetincr\onslide<.->
  text on slides 1-\\
  \onslide<+->
  text on slides 2-
\end{frame}
\end{verbatim}
The first call to |\onslide<+->| doesn't advance the slide, unless it has been
preceded by a |\resetincr| (or another |\onslide<+->| or a |\pause|).
\end{command}

\begin{command}{\fromincr\sarg{incr}}
  This is shorthand for
\begin{verbatim}
\resetincr[incr]
\onslide<.->
\end{verbatim}
It can only be used as a declaration (not with an argument).  The restrictions
on \meta{incr} are the same as above.  
\end{command}


\section{Labelling and referring to increments}\label{sec:labels}

In complicated frames, it may be useful to name certain increments for later
reference.  For instance, one might want to change a figure at certain steps
while progressing through a list of bullet points.  Or one might want to
redisplay certain slides in the frame with |\afterframe| or |\handoutframe|
(described below).

\begin{command}{\incrlabel \marg{label}}
  This command attaches the current increment number to the label \meta{label}.
  Once defined, the labelled increment can be recovered in (almost) any overlay
  spec using the constructs discussed below.  The \meta{label} can contain most
  characters, but should not start with a `|.|'  or `|!|'.
\end{command}

\begin{command}{\incrref \marg{incr spec}}
  This command returns the increment number defined by increment specification
  \meta{incr spec} as described below.
\end{command}

\noindent
The general form of an increment specification is
\begin{command}{{increment specification}: |label|\opt{|(offset)|}}
  The label can be a string assigned by a call to |\incrlabel|, or the special
  label `|.|' which refers to the current increment (this is subtly different to
  the incremental overlay specification `|.|').  The \meta{offset}, if given, is
  added to the increment indicated by the label.  It can be negative.

  Increment specifications can be used as part of almost any overlay
  specification by enclosing them with slashes, e.g. |</mylabel(2)/>|.
  \footnote{I have tried to find a way
      to use a specification like |</!mat/->| to both select and reset an
      increment in the manner of \texttt{fromincr}, but thus far this has eluded my
      \LaTeX\ hacking skills}
\end{command}

An example:
\example
\begin{verbatim}
\begin{frame}[label=twolists]
  \resetincr
  \begin{center}
    Two lists \onslide<+->{in sync}\\
    \onslide<+->{with more material\\}
    \onslide<+->{at the top}
  \end{center}
  \begin{columns}
    \begin{column}{.2\textwidth}
      \incrlabel{startlist}%
      \begin{itemize}[<+-| alert@+>]
      \item Apple \item Peach \incrlabel{halfway} \item Plum \item Orange
      \end{itemize}
    \end{column}
    \begin{column}{.2\textwidth}
      \resetincr[/startlist/]% keep in sync, even if we add extra topmatter
      \begin{itemize}[<+-| alert@+>]
      \item green \item yellow \item purple \item orange
      \end{itemize}
    \end{column}
  \end{columns}
  \vfill
  \onslide<+->
  The final increment is \incrref{.}. 
  \incrlabel{end}
\end{frame}
\end{verbatim}

Note that of commands discussed here, |\incrref| expects an increment
specification (i.e. |label(offset)|), while |\fromincr| and |\resetincr| expect
a restricted overlay specification that might include an increment specification
(e.g. |/label(offset)/|) or just a number.  Standard overlay-aware commands
should all accept overlay specifications that include increment specs.

One \beamer\ command (slightly patched in this package) with which named
increments are particularly useful is |\againframe|.  So \example
\begin{verbatim}
  \againframe<1,/halfway/,/end(-1)/-/end/>{twolists}
\end{verbatim}
provides an abbreviated tour of the lists.  Increment labels are associated with
the label of the enclosing frame, and so the same names can safely be reused
across multiple named frames.

There is also a similar new command called |\handoutframe| to render more than
one overlay from a frame in \texttt{handout} or similar modes that, by default,
just show a single slide with all the overlays collapsed.  See \cref{sec:helper}.


\section{\AmSTeX\ align environments}

The \texttt{align} and \texttt{align*} environments from the \texttt{amsmath}
package are useful for multiple or multiline equations, but don't really work
well with overlays.  The current package introduces a partial fix for this,
although their are remaining fragilities that need to be worked around.  It
should be straightforward to extend the approach used here to other
\texttt{amsmath} equation environments if desired.


\begin{environment}{{incralign[*]}\opt{\texttt{[<}\meta{spec1}\texttt{>\&<}\meta{spec2}\texttt{>\& ...]}}}
    Pre-process the input to |align[*]| to place an |\onslide<>{}| command
    around each field, defined as the material appearing between successive |&|
    and/or |\\| tokens.  By default, the first field on a line is called with
    |\onslide<+->{}|, and remaining ones with |\onslide<.->{}|.  This has the
    effect of displaying a full line at a time.  The optional argument makes it
    possible to change this to |\onslide<spec1>|, |\onslide<spec2>|, etc.\ with
    the sequence of specifications reset to |<spec1>| at the beginning of every
    line. There must be at least as many specifications given as their are
    fields used on the longest single line in the environment contents, but
    lines can have fewer fields.  The default specification values can be
    changed by a calling |\incraligndefaultspec| as below.

    It is also possible to override the default for a single field by starting
    it with the new specification.  This means that if the field content itself
    starts with |<|, it needs to be protected, e.g.\ with a leading |{}|.

    \example
\begin{verbatim}
  \begin{incralign*}[<+->&<.->&<+->&<.->] % both sides each eqns appear at once
    x\incrlabel{x} &= y  &  1 &{}< 2 \\
    </x/-> x^2 &= y^2  &  </x/->\resetincr[/x/] e^{i\pi} &<+->= -1 \\
    \sum_n f(n) &\to \int f(x) dx
  \end{incralign*}
  (increment at this line = \incrref{.})\\
\end{verbatim}
    
    The pre-processor is not able to distinguish between the |&| alignment
    characters that apply to the containing environment and any that appear
    within (e.g.) enclosed |array| environments.  So any such environments need
    to be protected using either a token register or a protected macro.  Note
    that it is still possible to use increments within the environments. These
    are processed sequentially with those in the containing |align| environment,
    respecting increment labels, resets etc.

    \example
\begin{verbatim}
  % using token registers
  \newtoks\mymatrix
  \mymatrix={\begin{pmatrix} 1 & 2 \\ \alt<+->{3}{2} & 4 \\ \end{pmatrix}}
  \begin{incralign}
    \incrlabel{mat}\the\mymatrix \resetincr[/mat/]& \text{is \only<+->{not }singular}
  \end{incralign}

  % using \protected
  \protected\def\mymatrix{\begin{pmatrix} 1 & 2 \\ \alt<+->{3}{2} & 4 \\ \end{pmatrix}}
  \begin{incralign}
    \incrlabel{mat}\mymatrix \resetincr[/mat/]& \text{is \only<+->{not }singular}
  \end{incralign}

\end{verbatim}
    It may be wise to put the |\newtoks| declaration outside the frame so as not
    to consume more of \TeX's resources than needed.  Also note that equation
    numbers aren't currently overlay aware.

    The current version does not support use of |\intertext|, but I hope to fix this.
\end{environment}



\begin{command}{\makealignincremental}
  Make all subsequent uses of the \texttt{align} and \texttt{align*}
  environments automatically incremental.  The original (non-incremental) forms
  are available as \texttt{amsalign[*]}.
\end{command}

\begin{command}{\makealignams}
  Make all subsequent uses of the \texttt{align} and \texttt{align*} invoke
  their original non-incremental forms.  
\end{command}

\begin{command}{\incraligndefaultspec\oarg{new default}}
  With the optional arg, change the default overlay specification for subsequent
  \texttt{incralign[*]} environments (and \texttt{align[*]} if
  |\makealignincremental| has been called).  Without the optional arg this
  restores the package default value which is
  |<+->&<.->&<.->&<.->&<.->&<.->&<.->&<.->|.
\end{command}









\section{Misc helper functions}\label{sec:helper}

These functions may prove useful under some circumstances.

\begin{command}{\handoutframe\oarg{mode spec}\sarg{overlay specification}\marg{frame label}}
  This command only produces output when compiled in \meta{mode spec} (which
  defaults to \texttt{handout}, but can include more than one, e.g.,
  \texttt{[beamer$\mid$handout]}), in which case it renders the specified
  overlays from the named frame.  If \meta{overlay specification} is omitted,
  all the overlays are rendered.  The code works by switching temporarily to
  \texttt{beamer} mode as that seems to be the only way to produce more than one
  overlay per frame in \texttt{handout}, \texttt{trans} and \texttt{article}
  modes, although this means it may behave poorly with any mode-specific
  material within the frame.  The idea is from
  \url{https://tex.stackexchange.com/questions/455444/beamer-overlays-and-handout-exclude-frames-from-handout/455459#455459}.
\end{command}

Unfortunately,  the natural code
\example
\begin{verbatim}
\begin{frame}<handout:0>[label=twolists]
  ...
\end{frame}
\handoutframe<1,/halfway/,/done/>{twolists}
\end{verbatim}
fails, because the |<handout:0>| spec stops the increment labels from being
defined.  If running a recent \LaTeX\ compiler (post 2021) the command
|\framescanonly<handout>| described below provides a workaround.
\example
\begin{verbatim}
\begin{frame}[label=twolists]
  \framescanonly<handout|trans>%
  ...
\end{frame}
\handoutframe[handout|trans]<1,/halfway/,/done/>{twolists}
\end{verbatim}


\begin{command}{\framescanonly\opt{\sarg{mode specification}}}
  Scan the current frame without producing any output.  This is similar to a
  |<mode:0>| specification to |\begin{frame}|, but as the frame is still scanned
  it allows side effects such as increment label definitions.  The |\frameonly|
  command should be placed inside the frame contents.  It is only available in
  recent versions of \LaTeX\ with hook support; a warning is printed in other
  cases.  If used in |beamer| mode the frame will be reprocessed for every
  overlay.  This behaviour can be avoided by also including a |<beamer:1>| or
  |<beamer:-1>| (but not |<beamer:0>|!) or equivalent specification to
  |\begin{frame}|.

  An example of its use appears above.
\end{command}

\begin{command}{\parseincrrefspec \marg{overlay spec}}
  Interpret any text enclosed in |/|s within \meta{overlay spec} as an increment
  specification, replacing them with the corresponding numerical values.
  \example
\begin{verbatim}
\begin{frame}
  \resetincr[2]
  \incrlabel{two}
  \parseincrrefspec{/two/-/two(3)/} % prints 2-5
\end{frame}
\end{verbatim}
  
\end{command}




\section{Internals}\label{sec:internals}

These sections discuss more background and some implementation details.   This
is only likely to be of interest to users who wish to extend the approach.

\subsection{Pauses and increments}

Both |\pause| and incremental overlay specifications access the same underlying
counter called |beamerpauses|, but they use them in different ways.

\begin{command}{\pause}
  increments |beamerpauses| and then sets subsequent material on the slide
  given by the incremented |\value{beamerpauses}|. 
\end{command}

\begin{command}{\onslide<+->}
  increments |beamerpauses| but then sets subsequent (or argument) material on
  the slide corresponding to the \emph{previous} value of |beamerpauses|.
\end{command}

\begin{command}{\onslide<.->}
  leaves |beamerpauses| alone, but sets subsequent (or argument) material on the
  slide given by |\value{beamerpauses}-1|, unless |\value{beamerpauses}=1| in
  which case it puts subsequent material on slide 1.
\end{command}

This conflict in interpretation of the |beamerpauses| counter can cause
unintuitive effects.  The incremental specfication model is far more flexible
and powerful, and so the commands of this package can all be interpreted in
terms of an \emph{increment number} which equals
|max(\value{beamerpauses}-1,1)|.  In fact, they all use the |beamerpauses|
counter with this offset.  Thus, when commands like |\resetincr| set the
increment value, they set |beamerpauses| to the increment + 1.  This value then
behaves sensibly with |<+>| etc.\ specifications, but not with |\pause|.  





  % for reference with \slideref{foo} (and/or </foo/>).
%
% The "increment number" is defined by the use of \pause and relative
% overlay commands like \onslide<+->.  In general, it corresponds to the slide
% on which text that followed the most recent <+> increment would
% appear.  For technical reasons this is one less than the value of the
% beamerpauses counter (unless that counter is 1).  This can cause confusion if
% \pause and <+> constructs are mixed.  

% the internal label name includes the frame label if one is given -- this helps
% to keep the references valid across frames



% \bibliographystyle{apalike} 
% \bibliography{journalsabbrv,neuro,learning}

\end{document}

% Local Variables:
% mode: latex
% eval: (set-fill-column 80)
% End:
